\documentclass[dvipdfmx]{jsarticle}
\title{Galois理論による代数学の基本定理の証明}
\author{MathTech部員 田中大地}
\date{}
\usepackage{graphicx}
\usepackage{amsmath}
\usepackage{amssymb}
\usepackage{amsthm}
\usepackage{bm}
\usepackage{mathtools}
\usepackage{tikz-cd}
\theoremstyle{definition}
\newtheorem{definition}{Definition}[section]
\newtheorem{prop}[definition]{Proposition}
\newtheorem{lemma}[definition]{Lemma}
\newtheorem{theorem}[definition]{Theorem}
\newtheorem{corollary}[definition]{Corollary}
\newtheorem{remark}[definition]{Remark}
\newtheorem{example}[definition]{Example}
\setlength{\parindent}{0pt}%インデント
\DeclareMathOperator{\Stab}{Stab}
\DeclareMathOperator{\Orb}{Orb}
\DeclareMathOperator{\Ker}{Ker}
\renewcommand{\qedsymbol}{$\blacksquare$}
\begin{document}
\maketitle
\section{はじめに}
こんにちは。今回は、Galois理論の話を書きたいと思います。
Galois理論は方程式の可解性や作図問題の応用がありますが、代数学の基本定理の証明にも
応用することができます。本記事でその証明を紹介したいと思います。
前提知識として群論(正規部分群、剰余群、準同型定理など)
と体論(分離拡大と正規拡大の基本性質など)を仮定するので、高校生には難しい
内容かと思いますが、大学の代数学の感じが伝わればいいなと思います。
まず、主張を正確に書きます。
\begin{theorem}\label{algethm}(代数学の基本定理)
 $ a_0,a_1,\ldots ,a_n \in \mathbb{C},a_n\ne 0$とする。\\
 このとき多項式$f(x)=a_nx^n+a_{n-1}x^{n-1}+\cdots +a_0 $は$\mathbb{C}$内に根を持つ。
\end{theorem}
証明は、代数学の基本定理が成り立たないと仮定すると複素数体$\mathbb{C}$の
二次拡大体が構成できるが、複素数体$\mathbb{C}$は二次拡大体を持たない事も示せるので、
矛盾となるというものです。
以下、証明に使う群論とガロア理論について説明します。
\section{群論}
群論で必要な定理は定理\ref{pgrp}と定理\ref{sylow}である。この二つの定理を
しめすために群の作用という概念を導入します。
\begin{definition}(群作用)
  $G$:群、$X$:集合とする。
  \[G\times X\ni (g,x)\longmapsto gx\in X\]
  がつぎの(1)、(2)を満たすとき、この写像を作用といい、
  $G\curvearrowright X$と書く。
  \begin{enumerate}
    \item[(1)] $e\in G$を単位元とすると、$\forall x\in X,ex=x$
    \item[(2)] $\forall g,h\in G,\forall x\in x,g(hx)=(gh)x$
  \end{enumerate}
 また、$G\curvearrowright X$の時、
 $x\in X$に対して、\\
 $\Stab(x)=\{g\in G\mid gx=g\}$を$x$の安定化群、
 $\Orb(x)=\{gx\mid g\in G\}$を$x$の軌道という。
\end{definition}
\newpage
次が成り立つ。
\begin{prop}(軌道分解)\qquad
  $G\curvearrowright X$のとき、$X=\bigsqcup  _{\Orb(x)}\Orb(x)$  
\end{prop}
\begin{proof}
  任意の$x\in X$に対して$x\in\Orb(x)$であり、また$\forall x,y\in X$に対して
  $\Orb(x)=\Orb(y)
  $または$\Orb(x)\cap\Orb(y)=\emptyset$
  なることが定義からわかるので、従う。
\end{proof}
\begin{prop}
  $G\curvearrowright X$とする。\\
  $x\in X$に対して、$\Stab(x)$は$G$の部分群であり、
  $\left|G/\Stab(x)\right|=\left|\Orb(x)\right|$となる。
\end{prop}
\begin{proof}
\begin{center}
    \begin{tikzcd}
      G/\Stab(x) \arrow[r] 
      & \Orb(x)  \\
      \bar{g} \arrow[r,mapsto] \arrow[u,phantom,"\in",sloped]
      & gx \arrow[u,phantom,"\in",sloped]
      \end{tikzcd}    
\end{center}
  によって一対一に対応する。
\end{proof}
\begin{definition}$G$を群とする。\\
    (1) $Z_G=\{h\in G\mid \forall g\in G,gh=hg\}$を$G$の中心という。\\
    (2) $p$を素数とするとき、位数が$p$べきである群を$p$群という。
\end{definition}
$G$:有限群、$X=G$として、
\[ G\times X\ni (g,h)\longmapsto ghg^{-1}\in X\]
を考えると作用になる。この作用について軌道分解の代表元$\{h_i\}$をとると、
\begin{align*}
             G=&\bigsqcup _{i}\left|\Orb(h_i)\right|\\
              =&\left(\bigsqcup _{i,h_i\in Z_G}\Orb(h_i)\right)\bigsqcup
              \left(\bigsqcup _{i,h_i\notin Z_G}\Orb(h_i)\right)\\
              =&\left(\bigsqcup _{i,h_i\in Z_G}\{h_i\}\right)\bigsqcup 
              \left(\bigsqcup _{i,h_i\notin Z_G}\Orb(h_i)\right)
\end{align*}
ここで、任意の$h\in Z_{G}$は
$\bigsqcup _{i,h_i\notin Z_G}\Orb(h)$
に含まれないから、$h\in\bigsqcup _{i,h_i\in Z_G}\{h_i\}$
従って$Z_{G}=\bigsqcup _{i,h_i\in Z_G}\{h_i\}$
となるから、
\begin{equation}\label{kyouyaku}
    |G|=|Z_{G}|
              +\sum_{i,h_i\notin Z_G}\left|\Orb(h_i)\right|
\end{equation}
$|G|=p^n$のとき、$h\notin Z_G$なら
$\left|\Orb(h)\right|
=\frac{|G|}{|\Stab(h)|}=\frac{p^n}{|\Stab(h)|},\>
\left|\Stab(h)\right|< |G|=p^n$となるから、$|\Orb(h)|$は$p$の倍数、
よって、(\ref{kyouyaku})より$|Z_G|$は$p$の倍数である。
また、$Z_G\supseteq \{e\}$だから、次が言えた。
\begin{theorem}
  $G$を$p$群とするするとき、$Z_G\supsetneq \{e\}$
\end{theorem}
\begin{theorem}\label{pgrp}
  $G$が$p$群で、$|G|=p^n$なら$0\leqq\forall t\leqq n$に対して
  位数$p^t$の部分群を持つ。
\end{theorem}
\begin{proof}
 $n$の帰納法で示す。$n=1$は明らか。$n-1$まで成り立つとして$n$のときを示す。
  $Z_G\varsupsetneq \{e\}$の位数pの元$x$をとり、$\mathrm{N}=\langle x\rangle$とする。
  $N$は$G$の正規部分群であり、$G/N$は位数$p^{n-1}$の群であるから帰納法の仮定から
  $0\leqq\forall t\leqq n-1$に対して指数$p^t$の部分群がある。言い換えれば、$G$には
  $0\leqq\forall t\leqq n-1$に対して指数$p^t$の部分群がある。ここで、
  部分群の対応
  \[\{G\mathrm{の}N\mathrm{を含む部分群}\}
  \longleftrightarrow
   \{G/N\mathrm{の部分群}\}\]
において、対応する部分群の指数は保たれる。実際、
$N\subset H\subset G$なる部分群Hをとると、
対応する$G/N$の部分群は$\tilde{H}=H/N$である。
このとき、つぎの写像
\[\left(G/N\right)/\tilde{H}\ni
\overline{g}\tilde{H}\longmapsto gH\in G/H\]
\[G/H\ni gH\longmapsto\overline{g}\tilde{H}\in\left(G/N\right)/\tilde{H}\]
は互いに逆写像となる。\\
よって、この部分群から$G$には指数が
$p^t(0\leqq t\leqq n-1)$の部分群が存在することが分かった。
$\{e\}$は指数$p^n$の部分群だから、$t=n$の場合も存在する。これは位数
$p^t(0\leqq \forall t\leqq n)$の部分群が存在することを意味するので
証明が完了した。
\end{proof}
\begin{theorem}\label{sylow}(Sylow)
  $G$を有限群で、$|G|=p^nm$($p,m$は互いに素)とする。このとき部分群$H$で
  $|H|=p^n$なるものがある。 
\end{theorem}
\begin{proof}
  $X=\{S\subset G\mid |S|=p^n(\mathrm{但し}S\mathrm{は部分群とは限らない})\}$とする。
\begin{center}
    \begin{tikzcd}
      G\times X \arrow[r] 
      & X  \\
      (g,s) \arrow[r,mapsto] \arrow[u,phantom,"\in",sloped]
      & gS=\{gs|s\in S\} \arrow[u,phantom,"\in",sloped]
      \end{tikzcd}  
\end{center}  
  は作用となる。
  \begin{align*}
                   |X|=&\left(
                              \begin{array}{c}
                                     p^nm\\
                                     p^n
                              \end{array}
                         \right)
                      =\left((x+1)^{p^nm}\mathrm{の}x^{p^n}\mathrm{の係数}\right) \\ 
    (x+1)^{p^nm}\equiv &(x^{p^{n}}+1)^m\qquad \mod p\\
    \therefore |X|=&\left(
                          \begin{array}{c}
                                  p^nm\\
                                  p^n
                          \end{array}
                    \right)
                  \equiv m\qquad \mod p
  \end{align*}
  軌道分解を考えると、\\
  \[|X|=\sum_{\Orb(S)}\left|\Orb(S)\right|\equiv m\not\equiv 0\qquad \mod p\]
  だから、$\left|\Orb(S)\right|\not\equiv 0$となる$S$がある。$H=\Stab(S)$
  とおく。
  \[S=\bigcup _{y\in H}Hy=\bigsqcup _{Hy}Hy\]
  よって$|Hg|=|H|$より$|S|=p^n$は$|H|$で割り切れるので、
  $|H|=p^k(0\leqq k\leqq n)$と書ける。また、
  \[\frac{|G|}{|H|}=p^{n-k}m
  =\left|\Orb(S)\right|\not\equiv0\qquad \mod p\]
  から、$k=n$である。よって$|H|=p^n$となり、$H$が求める部分群。
\end{proof}
\newpage
\section{Galois理論}
本節の目標はThm\ref{gal}である。
\begin{definition}
  $K/F$を代数拡大とする。\\
  (1) $\forall\alpha\in K$の$F$上最小多項式が$K$上で一次多項式の積に分解されるとき$K/F$を正規拡大という。\\
  (2) $\forall\alpha\in K$の$F$上最小多項式が$K$の代数閉包で重根を持たないとき$K/F$を分離拡大という。\\
  (3) $K/F$が正規拡大かつ分離拡大のとき、$K/F$をGalois拡大という。\\
  (4) $K/F$有限次Galois拡大とするとき
        $\mathrm{Gal}(K/F)=\{f\in \mathrm{Aut} (K)\mid f(x)=x(\forall x\in F)\}$をGalois群という。
\end{definition}
\begin{theorem}
  $K/F$を有限次Galois拡大とする。このとき、$[K:F]=\left|\mathrm{Gal}(K/F)\right|$
  ただし$[K:F]$は$K/F$の拡大次数。
\end{theorem}
\begin{proof}
  $K/F$は分離拡大だから、$\overline{K}$を$K$の代数閉包とするとき、
  $\left|\mathrm{Hom}_F(K,\overline{K})\right|=[K:F]$である。
  $K/F$は正規拡大でもあるから、$\forall \sigma\in \mathrm{Hom}_F(K,\overline{K})$について
  $\sigma (K)\subset K$である。$\Ker \sigma =0,[K:F]<\infty$
  なので次元定理から$\sigma(K)=K$となる。
  よって$\mathrm{Hom}_F(K,\overline{K})=\mathrm{Gal}(K/F)$なので、従う。
\end{proof}
\begin{theorem}\label{artin}(Artin)
  $K$を体とし、$G$を$\mathrm{Aut}(K)$の有限部分群とする。このとき\\
  $F=K^G=\{x\in K|\sigma (x)=x(\forall\sigma\in G)\}$
  とすれば、$K/F$は有限次Galois拡大で$\mathrm{Gal}(K/F)=G$となる。
\end{theorem}
\begin{proof}
  $\alpha\in K$に対し
  $\alpha =\{\sigma\in G|\sigma (\alpha)=\alpha\},
  G=\bigsqcup_{i=1}^{n} {\sigma_i H_\alpha}$,
  \[f_\alpha(X)=(X-\sigma _1(\alpha))(X-\sigma _2(\alpha))\cdots
  (X-\sigma _n(\alpha))\in K[X]\]
  とする。
  $\forall\sigma\in G$に対して
\begin{center}
    \begin{tikzcd}
      G/H_{\alpha} \arrow[r] 
      & G/H_{\alpha}  \\
      \sigma_{i}H_{\alpha} \arrow[r,mapsto] \arrow[u,phantom,"\in",sloped]
      & \sigma\sigma_{i}H_{\alpha} \arrow[u,phantom,"\in",sloped]
      \end{tikzcd}  
\end{center}
    は全単射であるので、
  \begin{align*}
    \sigma(f_\alpha(X))=&(X-\sigma\sigma _1(\alpha))(X-\sigma\sigma _2(\alpha))\cdots
    (X-\sigma\sigma _n(\alpha))\\
                =&f_\alpha(X)
  \end{align*}
  となり、$f(X)\in F[X]$。また、
  $\sigma _i(\alpha)=\alpha$なる$i$があるから、$f_\alpha(\alpha)=0$であり、
  $\sigma _i(\alpha)=\sigma _j(\alpha)\Longleftrightarrow 
  \sigma _i H_\alpha =\sigma _i H_\alpha$
  だから、$f_\alpha(X)$は分離多項式。よって$K/F$はGalois拡大。
  $K/F$が無限次拡大なら、$\beta _1,\beta _2,\ldots$を
  \[F\subsetneq F(\beta _1)\subsetneq  F(\beta _1,\beta _2)\subsetneq\cdots\]
  となるように取れるので、中間体
  $M(F\subset M\subset K)$で、
  $|G|<[F:M]<\infty$である$M$がある。$M/F$は有限次分離拡大だから、
  $M=F(\beta)$となる$\beta\in M$をとると
  \[[M:F]=\deg f_\beta(X)<|G|\]
  なるから、矛盾。$K/F$は有限次Galois拡大である。\\
  最後に、$\mathrm{Gal}(K/F)=G$を示す。定義より
  $\mathrm{Gal}(K/F)\supset G$である。
  $\forall \sigma\in \mathrm{Gal}(K/F)$に対して、
  \[f_\alpha (\sigma(\alpha))=\sigma (f_\alpha (\alpha))=0\]
  なので、$\sigma (\alpha)=\sigma _i(\alpha)(\exists i)
  \therefore \sigma =\sigma _i(\exists i)$
  よって、$\mathrm{Gal}(K/F)\subset G$となるから、
  $\mathrm{Gal}(K/F)=G$を得る。
\end{proof}
$K/F$を有限次Galois拡大、$G=\mathrm{Gal}(K/F)$とする。
\begin{itemize}
  \item 中間体$M$に対し、$H(M)=\{\sigma\in G\mid \forall x\in M,\sigma(x)=x\}$とおく。
  \item 部分群$H\subset G$に対して、$M_ H=\{x\in K\mid \forall\sigma\in H,\sigma(x)=x\}$とおく。
\end{itemize}
\begin{prop}\label{tyuukanngal}
  $K/F$を有限次Galois拡大とする。中間体$M$に対し、$K/M$有限次Galois拡大であり、
  $\mathrm{Gal}(K/F)=H(M)$となる。
\end{prop}
\begin{proof}
  定義から有限次Galois拡大であること、
  $\mathrm{Gal}(K/F)=H(M)$であることが確かめられる。
\end{proof}
\begin{theorem}\label{gal} (Galoisの基本定理)\\
  $K/F$を有限次Galois拡大とする。
  \begin{enumerate}
    \item [(1)]
          $\mathbb{M}=\{K/F\mathrm{の中間体}M\},\mathbb{H}=\{\mathrm{Gal}(K/F)\mathrm{の部分群}\}$
          とおくとき、
          \[\mathbb{M}\ni M\longmapsto H(M)\in\mathbb{H}\]
          \[\mathbb{H}\ni H\longmapsto M_H\in\mathbb{M}\]
           によって、一対一に対応する。
    \item [(2)]
          $M_1,M_2\in\mathbb{M}$がそれぞれ$H_1,H_2\in\mathbb{H}$に対応するとき
          \begin{align*}
            M_1\subset M_2&\Longleftrightarrow H_1\supset H_2\\
            M_1\cdot M_2&\longleftrightarrow H_1\cap H_2\\
            M_1\cap M_2&\longleftrightarrow \langle H_1,H_2\rangle
          \end{align*}
    \item[(3)]
         $M\longleftrightarrow H$のとき
         \[M/F\mathrm{がGalois拡大}\Longleftrightarrow H\vartriangleleft G\]
         であり、このとき
         \[\mathrm{Gal}(M/F)\cong \mathrm{Gal}(K/F)/H\]
  \end{enumerate}
\end{theorem}
\begin{proof}
  (1)
$M\in\mathbb{M}$とする。$M_{H(M)}=\{x\in K\mid \forall \sigma\in H(M),\sigma (x)=x\}$
であるから、$M\subset M_{H(M)}$である。Thm\ref{artin}より$K/M_{H(M)}$は有限次Galois拡大
であり、$\mathrm{Gal}(K/M_{H(M)})=H(M)$となる。一方Prop\ref{tyuukanngal}
から$K/M$も有限次Galois拡大で$\mathrm{Gal}(K/M)=H(M)$なので、
\[[M_{H(M)}:M]=\frac{[K:M]}{[K:M_{H(M)}]}=\frac{|H(M)|}{|H(M)|}=1\]
よって$M=M_{H(M)}$。\\
$H\in\mathbb{H}$とする。$H(M_H)=\{\sigma\in \mathrm{Gal}(K/F)|\forall x\in M_H,\sigma(x)=x\}$
であるので、$H\subset H(M_H)$となる。Thm\ref{artin}から
$K/M_H$は有限次Galois拡大であり、$\mathrm{Gal}(K/M_H)=H$となる。一方Prop\ref{tyuukanngal}
より$\mathrm{Gal}(K/M_H)=H(M_H)$なので、$H=H(M_H)$となる。\\
(2)
$M_1\subset M_2\Longleftrightarrow H_1\supset H_2$は明らか。\\
$M_1\cdot M_2\supset M_i(i=1,2)$より、$H(M_1\cdot M_2)\subset H(M_1)\cap H(M_2)$
である。$ H(M_1)\cap H(M_2)$の各元は$M_1,M_2$を不変にするので、$M_1\cdot M_2$を不変にする。
よって$H(M_1\cdot M_2)\supset H(M_1)\cap H(M_2)$だから、$H(M_1\cdot M_2)= H(M_1)\cap H(M_2)$\\
$M_1\cap M_2\subset M_i(i=1,2)$なので、$H(M_1\cap M_2)\supset \langle H(M_1),H(M_2)\rangle= \langle H_1,H_2\rangle $
である。$H=\langle H_1,H_2\rangle$とすると、$H\supset H_i(i=1,2)$だから、
$M_H\subset M_{H_i}=M_i(i=1,2)$となり、$M_H\subset M_1\cap M_2
\therefore H\supset H(M_1\cap M_2)$よって
$H(M_1\cap M_2)=\langle H_1,H_2\rangle$\\
(3)
$M$を中間体とする。$M\longleftrightarrow H$とする。
\begin{align*}
  M/F\text{が有限次Galois拡大}&\xLeftrightarrow{(a)} M/F\text{が正規拡大}\\
                               &\xLeftrightarrow{(b)} \forall\sigma\in\mathrm{Gal}(K/F),\sigma (M)\subset M\\
                               &\xLeftrightarrow{(c)} H\vartriangleleft \mathrm{Gal}(K/F)
\end{align*}
$(a)$
$K/F$が有限次分離拡大だから成り立つ。\\
$(b)$
$(\Rightarrow)\;\>\alpha\in M$の$F$上最小多項式を$f(X)$とすると、$M/F$が正規拡大であることから
$f(X)=\prod _i (X-\alpha_i)\quad (\alpha_i\in M)$とかける。$f(\sigma (\alpha))=\sigma (f(\alpha))=0$
なので、 $\sigma (\alpha)=\alpha _i\in M$である。\\
$(\Leftarrow)\;\>\>\alpha\in M$とする。
\[f(X)=\prod _{\sigma\in\mathrm{Gal}(K/F)}(X-\sigma (\alpha))\]
とすると、$\sigma(f(X))=f(X)\>(\forall \sigma)$より$f(X)\in F[X]$である。
$f(X)$は$\alpha$を根に持ち$M$上で一次因子の積に分解されるので、$\alpha$の$F$上
最小多項式も$M$上で一次因子の積に分解されるので$M/F$は正規拡大である。\\
(c)
$(\Rightarrow)\;\>$$\sigma\in\mathrm{Gal}(K/F),\tau\in H,x\in M$に対して
$\sigma(x)\in M$ゆえ$\sigma ^{-1}\tau\sigma (x)=\sigma ^{-1}\sigma (x)=x$となるから
$\sigma ^{-1}\tau\sigma\in H$よって$H\vartriangleleft\mathrm{Gal}(K/F)$となる。\\
$(\Leftarrow)\;\>$$\sigma\in\mathrm{Gal}(K/F),\tau\in H$について
$\sigma ^{-1}\tau\sigma\in H$なので$x\in M$なら
$\sigma ^{-1}\tau\sigma(x)=x$つまり$\tau\sigma(x)=\sigma(x)$
これが任意の$\tau\in H$で成り立つので$\sigma(x)\in M$よって、
$\sigma(M)\subset M$\\
最後に$\mathrm{Gal}(M/F)\cong \mathrm{Gal}(K/F)/H$について\\
\[
\mathrm{Gal}(K/F)\ni \sigma\mapsto \sigma|_{M}\in \mathrm{Gal}(M/F)
\]
に準同型定理を使うと、単射準同型
\[
\mathrm{Gal}(K/F)/H\ni \sigma H\mapsto \sigma|_{M}\in \mathrm{Gal}(M/F)
\]を得る。
ここで、Prop\ref{tyuukanngal}より、$H=\mathrm{Gal}(K/M)$なので
\begin{align*}
  \left|\mathrm{Gal}(K/F)/H\right|&=\frac{|\mathrm{Gal}(K/F)|}{|\mathrm{Gal}(K/M)|}\\
                                  &=\frac{[K:F]}{[K:M]}\\
                                  &=[M:F]\\
                                  &=|\mathrm{Gal}(M/K)|
\end{align*}
となるから、全射でもあるから$\mathrm{Gal}(M/F)\cong \mathrm{Gal}(K/F)/H$がいえた。
\end{proof}
\newpage
本記事の目標には関係ないが、興味深いのでGalois群の計算例を紹介しよう。
\begin{example}\label{ex1}
  体の拡大$\mathbb{Q}(\sqrt[4]{2},i)/\mathbb{Q}$を考える。
  これは、$f(X)=X^4-2\in \mathbb{Q}[X]$の最小分解体だから、有限次Galois拡大
  である。Eisensteinの既約判定定理から$f(X)$は既約多項式であるから
  $[\mathbb{Q}(\sqrt[4]{2}]:\mathbb{Q}]=4$である。
  $i$の$\mathbb{Q}$上最小多項式は$X^2+1$であるので、
  $[\mathbb{Q}(\sqrt[4]{2},i):\mathbb{Q}(\sqrt[4]{2})]\leqq 2$である。
  また$i\not\in\mathbb{Q}$なので、
  $[\mathbb{Q}(\sqrt[4]{2},i):\mathbb{Q}(\sqrt[4]{2})]=2$となる。
  ゆえに、$[\mathbb{Q}(\sqrt[4]{2},i):\mathbb{Q}]=8$である。Galois群
  を$G$とおく。
  $\sigma\in G$は、
  $(\sigma(\sqrt[4]{2}),\sigma(i))\in
  \{\sqrt[4]{2},-\sqrt[4]{2},i\sqrt[4]{2},-i\sqrt[4]{2}\}\times 
  \{i,-i\}$により決まるが、$|G|=8$だから、どの8パターンの対応をとる
  $\sigma\in G$も存在する。ここで
  \[\alpha _1=\sqrt[4]{2},\alpha _2=-\sqrt[4]{2},
  \alpha _3=i\sqrt[4]{2},\alpha_4=-i\sqrt[4]{2}\]
  とおく。すると次の対応を定める単射準同型$\phi$が考えられる。
  \begin{center}
     \begin{tikzcd}    
    G \arrow[r] 
    & S_{4}  \\
    \sigma \arrow[r,mapsto] \arrow[u,phantom,"\in",sloped]
    & \begin{pmatrix}
        1&2&3&4\\
        t_1&t_2&t_3&t_4
        \end{pmatrix} \arrow[u,phantom,"\in",sloped]
    \end{tikzcd}   
  \end{center}
 
ただし$\sigma(\alpha_i)=\alpha_{t_i}(i=1,2,3,4)$である。\\
$\sigma_1,\sigma_2\in G$を$\sigma_1(\alpha_1)=\alpha_3,\sigma_1(i)=i,
\sigma_2(\alpha_1)=\alpha_1,\sigma_2(i)=-i$となるようにとるとき、
$G$は$\sigma_1,\sigma_2$で生成される。
$\phi(\sigma_1)=\begin{pmatrix}1&3&2&4\end{pmatrix}$,
$\phi(\sigma_2)=\begin{pmatrix}3&4\end{pmatrix}$
であるので、$G\cong  \left\langle 
  \begin{pmatrix}1&3&2&4\end{pmatrix},
  \begin{pmatrix}3&4\end{pmatrix}
\right\rangle $となる。
\end{example}
\begin{example}\label{xe2}(Artin-Schreier拡大)\\
 $\mathrm{ch}F=p>0$なる体$F$と$F$上多項式$f(X)=X^p -X-a$を考える。
 ここで$a\notin\{x^p-x\mid x\in F\}$ とする。$f(X)$の根の一つ$\alpha$
 を添加した体を$E=F(\alpha)$としたとき$E/F$が$p$次Galois拡大となることを示そう。\\
 まず、$\alpha$の最小多項式を$g(X)$とするとき、$g(X)=g(X+1)$となることを背理法で
 示す。$g(X)\ne g(X+1)$を仮定すると、$g(X),g(X+1),\ldots ,g(X+p-1)$は相異なる。
 実際、$g(X+i)=g(X+j)\quad (0\leqq i<j\leqq p-1)$となるなら、$g(X)=g(X+j-i)$
 である。ここで、$(j-i)k+pl=1$となる$k,l$をとると、
 \[g(X)=g(X+k(j-i))=g(X+1-pl)=g(X+1)\]
  となるので矛盾が起きる。
  また、$g(X),g(X+1),\ldots ,g(X+p-1)$は、それぞれ$f(X)$の根である$\alpha,\alpha +1,\ldots ,\alpha +p-1$
  の最小多項式であるので、$f(X)=\prod _{i=0}^{p-1}g(X+i)$となることから、
  $p=\deg f(X)=\sum _{i=0}^{p-1}\deg g(X+i)=p\deg g(X)$である。
  これは、$\alpha\in F$を意味するが、$a\notin\{x^p-x\mid x\in F\}$に反するので
  $g(X)=g(X+1)$が示された。\\
  $g(X)$は相異なる$p$個の根$\alpha ,\alpha+1,\ldots ,\alpha +p-1$をもつので
  $g(X)=f(X)\prod _{i=0}^{p-1}(X-i)$である。以上から、
  $E/F$は$p$次Galois拡大である。したがって、
  $\mathrm{Gal}(E/F)\cong \mathbb{Z}/p\mathbb{Z}$である。
\end{example}
\newpage
\section{代数学の基本定理の証明}
証明に使う命題をいくつか準備する。
\begin{prop}\label{fact}
  $\mathbb{R}$上の任意の奇数次数多項式は、$\mathbb{R}$内に根を持つ。
\end{prop}
\begin{proof}
  中間値の定理から従う。
\end{proof}
補題として次の二つを示す。
\begin{lemma}\label{kannzennsei}
 標数0の体$F$の任意の代数拡大は分離拡大である。
\end{lemma}\label{nijikakudainohukanousei}
\begin{proof}
  $K$を$F$の代数拡大体とする。$\alpha\in K$の$F$上最小多項式を$f(X)$とする。
  $F$の標数0だから$\mathrm{deg}\frac{df}{dX}(X)=\mathrm{deg}f(X)-1$となる。
  $f(X)$が分離多項式でないなら、$f(X)$と$\frac{df}{dX}(X)$は共通根$\beta$を持つ。
  これは、$f(X)$が$\beta$の最小多項式であることに矛盾するから、$f(X)$は分離多項式である。
  したがって、$K/F$は分離拡大。
\end{proof}
\begin{remark}
実数体$\mathbb{R}$は標数0である。Thm\ref{algethm}の証明では、$\mathbb{R}$のGalois拡大体
を構成するが、Lem\ref{kannzennsei}によって分離性はただちに従うから正規性だけが問題になる。

\end{remark}
\begin{lemma}
  複素数体$\mathbb{C}$は二次拡大体を持たない。
\end{lemma}
\begin{proof}
$L$を$\mathbb{C}$の二次拡大体とする。
このとき、$\forall\alpha\in L\setminus \mathbb{C}$はあるモニック二次多項式の根だから、
解の公式から
\[\alpha=\frac{a\pm \sqrt{b}}{2}\qquad(a,b\in\mathbb{C})\]
と書ける。右辺は、$\mathbb{C}$の元だから、$L=\mathbb{C}$となる。
これは、二次拡大であることに矛盾する。
\end{proof}
準備ができたので代数学の基本定理を証明する。
\begin{proof}(Thm\ref{algethm}の証明)
代数学の基本定理が成り立たないとすると、$\mathbb{C}$上の既約多項式$f(X)$で、
次数が2以上のものがある。$K=\mathbb{C}[X]/(f(X))$とすれば、$K$は$\mathbb{C}$の有限次拡大
であり、拡大次数は2以上である。 $\mathbb{C}$は$\mathbb{R}$の二次拡大体であるから、
$K/\mathbb{R}$は有限次拡大である。Lem\ref{kannzennsei}より、$K/\mathbb{R}$
は有限次分離拡大である。$K$の$\mathbf{R}$に関するGalois閉包を$\tilde{K}$とする。
つまり、$K$の各元の$\mathbb{R}$上の共役元をすべて$\mathbb{R}$に添加した体とする。
$K\subset\tilde{K}$であり、$\tilde{K}/\mathbb{R}$は有限次Galois拡大となる。このとき、
$\tilde{K}/\mathbb{R}$の拡大次数が2のべきとなる。\\
$[\tilde{K}:\mathbb{R}]=2^nm(m\mathrm{は奇数})$とおくとき、
Thm\ref{sylow}よりGalois群$G$の部分群$H$で$|H|=2^n$となるものが取れる。
$H$に対応する。中間体$M(\mathbb{R}\subset M\subset \tilde{K})$をとると、
\[[M:\mathbb{R}]=\frac{[\tilde{K}:\mathbb{R}]}{[\tilde{K}:M]}
                =\frac{|\mathrm{Gal}(\tilde{K}/\mathbb{R})|}{|\mathrm{Gal}(\tilde{K}/M)|}
                \overset{\mathrm{Thm}\ref{tyuukanngal}}{=}\frac{|G|}{|H|}
                =m\]
$M=\mathbb{R}(\alpha)$なる$\alpha$をとると、$\alpha$の最小多項式$g(X)\in\mathbb{R}$
の次数は$m$であるので、$g(X)$の既約性からProp\ref{fact}より、$m=1$でなくてはならない。
よって$[\tilde{K}:\mathbb{R}]=2^n$となる。$[\tilde{K}:\mathbb{C}]$は
$[\tilde{K}:\mathbb{R}]=2^n$の約数だから$[\tilde{K}:\mathbb{C}]=2^l,l>0$となる。
Prop\ref{tyuukanngal}より、$\tilde{K}/\mathbb{C}$は有限次Galois拡大である。
Galois群の位数は$2^l$だから、Prop\ref{pgrp}から、位数$2^{l-1}$の
部分群$N\subset \mathrm{Gal}(\tilde{K}/\mathbb{C})$をとると、
$N$に対応する$\tilde{K}/\mathbb{C}$中間体は$\mathbb{C}$の二次拡大体である。
これは、$\mathbb{C}$の二次拡大体が存在しないことに矛盾するから、
代数学の基本定理は成り立つ。
\end{proof}
\newpage
\section{さいごに}
Thm\ref{gal}は、有限次Galois拡大$K/F$の構造をGalois群が支配しているという
群と体をつなぐ興味深い定理です。今回は、かなり抽象的に使ったが、Example\ref{ex1}のように
具体的に与えられた体の拡大のGalois群を計算するのも面白いです。他にも、$\mathbb{Q}$
に$\zeta _n=\exp (2\pi i/n)\in\mathbb{C}$を添加した体$\mathbb{Q}(\zeta_n)$
を考えると$\mathbb{Q}(\zeta _n)/\mathbb{Q}$は、有限次Galois拡大になっていて
Galois群が$\left(\mathbb{Z}/n\mathbb{Z}\right)^{\times}$に同型になること
がよく知られています。(このような拡大を円分拡大といいます。)

\begin{thebibliography}{10}
 \bibitem{yukie1}
 \newblock 雪江明彦,代数学1,日本評論社,2010
 \bibitem{yukie2}
 \newblock 雪江明彦,代数学2,日本評論社,2010
 \bibitem{hujisai}
 \newblock 藤崎源二郎,体とガロア理論,岩波書店,1991
 \bibitem{artin}
 \newblock E.Artin 訳寺田文行,ガロア理論入門,ちくま学芸文庫,2010
\end{thebibliography}
\end{document}